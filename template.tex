% FortySecondsCV LaTeX template
% Copyright © 2019-2020 René Wirnata <rene.wirnata@pandascience.net>
% Licensed under the 3-Clause BSD License. See LICENSE file for details.
%
% Please visit https://github.com/PandaScience/FortySecondsCV for the most
% recent version! For bugs or feature requests, please open a new issue on
% github.
%
% Contributors
% ------------
% * ifokkema
% * Bertbk
% * Hespe
%
% Attributions
% ------------
% * fortysecondscv is based on the twentysecondcv class by Carmine Spagnuolo
%   (cspagnuolo@unisa.it), released under the MIT license and available under
%   https://github.com/spagnuolocarmine/TwentySecondsCurriculumVitae-LaTex
% * further attributions are indicated immediately before corresponding code


%-------------------------------------------------------------------------------
%                             ADDITIONAL PACKAGES
%-------------------------------------------------------------------------------
\documentclass[
	a4paper,
	% showframes,
	% vline=2.2em,
	% maincolor=cvgreen,
	% sidecolor=gray!50,
	% sectioncolor=red,
	% subsectioncolor=orange,
	% itemtextcolor=black!80,
	% sidebarwidth=0.4\paperwidth,
	% topbottommargin=0.03\paperheight,
	% leftrightmargin=20pt,
	% profilepicsize=4.5cm,
	% profilepicborderwidth=3.5pt,
	% profilepicstyle=profilecircle,
	% profilepiczoom=1.0,
	% profilepicxshift=0mm,
	% profilepicyshift=0mm,
	% profilepicrounding=1.0cm,
]{fortysecondscv}

% improve word spacing and hyphenation
\usepackage{microtype}
\usepackage{ragged2e}

% uncomment in case you don't want any hyphenation
% \usepackage[none]{hyphenat}

% take care of proper font encoding
\ifxetexorluatex
	\usepackage{fontspec}
	\defaultfontfeatures{Ligatures=TeX}
%	\newfontfamily\headingfont[Path = fonts/]{segoeuib.ttf} % local font
\else
	\usepackage[utf8]{inputenc}
	\usepackage[T1]{fontenc}
%	\usepackage[sfdefault]{noto} % use noto google font
\fi

% enable mathematical syntax for some symbols like \varnothing
\usepackage{amssymb}

% bubble diagram configuration
\usepackage{smartdiagram}
\smartdiagramset{
	% default font size is \large, so adjust to harmonize with sidebar layout
	bubble center node font = \footnotesize,
	bubble node font = \footnotesize,
	% default: 4cm/2.5cm; make minimum diameter relative to sidebar size
	bubble center node size = 0.4\sidebartextwidth,
	bubble node size = 0.25\sidebartextwidth,
	distance center/other bubbles = 1.5em,
	% set center bubble color
	bubble center node color = maincolor!70,
	% define the list of colors usable in the diagram
	set color list = {maincolor!10, maincolor!40,
	maincolor!20, maincolor!60, maincolor!35},
	% sets the opacity at which the bubbles are shown
	bubble fill opacity = 0.8,
}

\hypersetup{
    colorlinks=true,
    urlcolor=blue,
    pdftitle={CV Michael Planu},
    pdfauthor={Michael Planu}
}

%-------------------------------------------------------------------------------
%                            PERSONAL INFORMATION
%-------------------------------------------------------------------------------
%% mandatory information
% your name
\cvname{Michael Planu}
% job title/career
\cvjobtitle{\faGraduationCap BSC - Computer Science}

%% optional information
% profile picture
\cvprofilepic{pics/profile.jpg}

% NOTE: ordering in sidebar will mimic the following order
% date of birth
\cvbirthday{August 27, 1994}
% short address/location, use \newline if more than 1 line is required
\cvaddress{Uta (CA), Italy}
% phone number
\cvphone{+39 3409675287}
% email address
\cvmail{michael.planu@icloud.com}


%-------------------------------------------------------------------------------
%                              SIDEBAR 1st PAGE
%-------------------------------------------------------------------------------
% add more profile sections to sidebar on first page
\addtofrontsidebar{
	% include gosquare national flags from https://github.com/gosquared/flags;
	% naming according to ISO 3166-1 alpha-2 country codes
	\graphicspath{{pics/flags/}}
	
    \profilesection{About Me}
    % \aboutme{Mi definisco un programmatore pragmatico [cit.]\\ Sono sempre stato una persona molto curiosa, tratto che mi porta ad avere conoscenze molto orizzontali ed essere sempre alla ricerca di miglioramenti, sia in ambito lavorativo che personale.  }

    \aboutme{I consider myself a pragmatic programmer \href{https://en.wikipedia.org/wiki/The_Pragmatic_Programmer}{[cit.]}\\ I have always been a very curious person, which leads me to have a wide range of knowledge and constantly seek improvements, both in my professional and personal life.}
	
	\profilesection{Languages}
	\pointskill{\flag{IT.png}}{Italian}{5}
	\pointskill{\flag{GB.png}}{English (B2) }{4}
	
	% social network accounts incl. proper hyperlinks
	\profilesection{Social Network}
		\begin{icontable}{2.5em}{1em}
			\social{\faLinkedin}
			    {https://www.linkedin.com/in/michael-planu}
			    {Linkedin}
			\social{\faGithub}
				{https://github.com/ilMike42}
				{Github}
		\end{icontable}



	\profilesection{Hobbies}
	    \skill{\faMicrochip}{Electronics and home automation}
		\skill{\faGamepad}{Gaming}
		\skill{\faTerminal}{Computer science}

	\profilesection{Soft Skills}
		\pointskill{\faUserFriends}{Teamwork}{5}[5]
			\skill[1.8em]{\faBriefcase}{Experience within a team in a corporate environment}
			\skill[1.8em]{\faGraduationCap}{University projects}
		\pointskill{\faBrain}{Problem solving}{5}[5]
}

\begin{document}

\makefrontsidebar

\cvsection{Work experience}
\begin{cvtable}[2.3]
    \cvitem{2018 - Oggi}
        {Mobile and web developer}
        {Softfobia s.r.l.}
        {Developed and maintained various mobile and web apps for projects of different sizes and for clients of different types. In these projects, I often had the opportunity to work using Agile methodology, with constant communication with the client. In the mobile field, I have worked on both native frameworks for Android and iOS, as well as utilizing hybrid technologies such as Ionic or Flutter, always striving to maintain a critical eye on the most suitable architectural patterns for each individual case.}
\end{cvtable}


\cvsection{Education}
\begin{cvtable}[3]
	\cvitem{2015 - 2019}
        {BSC - Computer Science}
        {University of Cagliari}
        {Thesis related to the field of sentiment analysis. Correlation between the sentiment of Stocktwits users, obtained from tweet analysis, and the actual performance of the SP500 index.}
    \cvitem{2014}
        {High School diploma in Technical Expertise, specializing in Computer Science}
        {ITIS Giua}
        {\\}
\end{cvtable}

\cvsection{Technical skills}
\begin{cvtable}[2.6]
    \cvitem{\texttt{UIKit\\ SwiftUI}}
        {iOS Apps}
        {}
        {Development and maintenance of apps, implementation of libraries, frameworks, and SDKs. Good knowledge of the most famous architectural patterns, UIKit and SwiftUI, and Combine.}

    \cvitem{\texttt{Angular\\ Ionic\\HTML5+CSS3}}
        {Mobile/Web frontend}
        {}
        {Technologies used across multiple projects for the creation and maintenance of web and mobile apps, also utilizing various third-party plugins.\\ Technology used to develop \href{https://github.com/ilMike42/portfolio-full-stack-monorepo}{my website}.}
        
    \cvitem{\texttt{Flutter\\ Dart}}
        {Frontend mobile app}
        {}
        {Development and maintenance of apps, implementation of libraries, frameworks, and SDKs.}
    
        \cvitem{\texttt{Java\\ Kotlin}}
        {Android Apps}
        {}
        {Development and maintenance of apps, implementation of libraries, frameworks, and SDKs.}
    
    \cvitem{\texttt{Python}}
        {Script for collecting and analyzing data related to the thesis, personal projects.}
        {}
        {Analysis and representation of data based on financial tweets, collected through the StockTwits API. The language is also used to create bots on Telegram and other side projects.}
    
    \cvitem{\texttt{PHP}}
        {Website management}
        {}
        {On a professional level: development and maintenance of projects based on Drupal. Also used during my academic career.}
    
    \cvitem{\texttt{C\# + Unity}}
        {Video Game Design exam}
        {}
        {Developed, as a team, a first-person 3D puzzle game based on survival and puzzle-solving to progress through levels.}
    
    \cvitem{Altri}
        {C, C++, YAML, LaTex, SQL, NoSQL}
        {}
        { }
\end{cvtable}

\cvsignature

\let\thefootnote\relax\footnotetext{In compliance with the Italian legislative Decree no. 196 dated 30/06/2003, I hereby authorize you to use and process my personal details contained in this document.}

\end{document}