% FortySecondsCV LaTeX template
% Copyright © 2019-2020 René Wirnata <rene.wirnata@pandascience.net>
% Licensed under the 3-Clause BSD License. See LICENSE file for details.
%
% Please visit https://github.com/PandaScience/FortySecondsCV for the most
% recent version! For bugs or feature requests, please open a new issue on
% github.
%
% Contributors
% ------------
% * ifokkema
% * Bertbk
% * Hespe
%
% Attributions
% ------------
% * fortysecondscv is based on the twentysecondcv class by Carmine Spagnuolo
%   (cspagnuolo@unisa.it), released under the MIT license and available under
%   https://github.com/spagnuolocarmine/TwentySecondsCurriculumVitae-LaTex
% * further attributions are indicated immediately before corresponding code


%-------------------------------------------------------------------------------
%                             ADDITIONAL PACKAGES
%-------------------------------------------------------------------------------
\documentclass[
	a4paper,
	% showframes,
	% vline=2.2em,
	% maincolor=cvgreen,
	% sidecolor=gray!50,
	% sectioncolor=red,
	% subsectioncolor=orange,
	% itemtextcolor=black!80,
	% sidebarwidth=0.4\paperwidth,
	% topbottommargin=0.03\paperheight,
	% leftrightmargin=20pt,
	% profilepicsize=4.5cm,
	% profilepicborderwidth=3.5pt,
	% profilepicstyle=profilecircle,
	% profilepiczoom=1.0,
	% profilepicxshift=0mm,
	% profilepicyshift=0mm,
	% profilepicrounding=1.0cm,
]{fortysecondscv}

% improve word spacing and hyphenation
\usepackage{microtype}
\usepackage{ragged2e}

% uncomment in case you don't want any hyphenation
% \usepackage[none]{hyphenat}

% take care of proper font encoding
\ifxetexorluatex
	\usepackage{fontspec}
	\defaultfontfeatures{Ligatures=TeX}
%	\newfontfamily\headingfont[Path = fonts/]{segoeuib.ttf} % local font
\else
	\usepackage[utf8]{inputenc}
	\usepackage[T1]{fontenc}
%	\usepackage[sfdefault]{noto} % use noto google font
\fi

% enable mathematical syntax for some symbols like \varnothing
\usepackage{amssymb}

% bubble diagram configuration
\usepackage{smartdiagram}
\smartdiagramset{
	% default font size is \large, so adjust to harmonize with sidebar layout
	bubble center node font = \footnotesize,
	bubble node font = \footnotesize,
	% default: 4cm/2.5cm; make minimum diameter relative to sidebar size
	bubble center node size = 0.4\sidebartextwidth,
	bubble node size = 0.25\sidebartextwidth,
	distance center/other bubbles = 1.5em,
	% set center bubble color
	bubble center node color = maincolor!70,
	% define the list of colors usable in the diagram
	set color list = {maincolor!10, maincolor!40,
	maincolor!20, maincolor!60, maincolor!35},
	% sets the opacity at which the bubbles are shown
	bubble fill opacity = 0.8,
}

\hypersetup{
    colorlinks=true,
    urlcolor=blue,
    pdftitle={CV Michael Planu},
    pdfauthor={Michael Planu}
}

%-------------------------------------------------------------------------------
%                            PERSONAL INFORMATION
%-------------------------------------------------------------------------------
%% mandatory information
% your name
\cvname{Michael Planu}
% job title/career
\cvjobtitle{\faGraduationCap Laureato in Informatica}

%% optional information
% profile picture
\cvprofilepic{pics/profile.jpg}

% NOTE: ordering in sidebar will mimic the following order
% date of birth
\cvbirthday{27 Agosto 1994}
% short address/location, use \newline if more than 1 line is required
\cvaddress{Uta (CA), via Su Pixinali 64}
% phone number
\cvphone{+39 3409675287}
% email address
\cvmail{michael.planu@icloud.com}


%-------------------------------------------------------------------------------
%                              SIDEBAR 1st PAGE
%-------------------------------------------------------------------------------
% add more profile sections to sidebar on first page
\addtofrontsidebar{
	% include gosquare national flags from https://github.com/gosquared/flags;
	% naming according to ISO 3166-1 alpha-2 country codes
	\graphicspath{{pics/flags/}}
	
    \profilesection{About Me}
    \aboutme{Mi definisco un programmatore pragmatico [cit.]\\ Sono sempre stato una persona molto curiosa, tratto che mi porta ad avere conoscenze molto orizzontali ed essere sempre alla ricerca di miglioramenti, sia in ambito lavorativo che personale.  }
	
	\profilesection{Lingue Conosciute}
	\pointskill{\flag{IT.png}}{Italiano}{5}
	\pointskill{\flag{GB.png}}{Inglese (B2) }{3}
	
	% social network accounts incl. proper hyperlinks
	\profilesection{Social Network}
		\begin{icontable}{2.5em}{1em}
			\social{\faLinkedin}
			    {https://www.linkedin.com/in/michael-planu}
			    {Linkedin}
			\social{\faGithub}
				{https://github.com/ilMike42}
				{Github}
		\end{icontable}



	\profilesection{Hobbies}
	    \skill{\faMicrochip}{Elettronica e domotica}
		\skill{\faGamepad}{Gaming}
		\skill{\faTerminal}{Informatica}

	\profilesection{Soft Skills}
		\pointskill{\faUserFriends}{Lavoro in Team}{5}[5]
			\skill[1.8em]{\faBriefcase}{Esperienza all'interno di un team in ambito aziendale}
			\skill[1.8em]{\faGraduationCap}{Progetti universitari}
		\pointskill{\faBrain}{Problem solving}{5}[5]
}

\begin{document}

\makefrontsidebar

\cvsection{Esperienze lavorative}
\begin{cvtable}[3]
    \cvitem{2018 - Oggi}
        {Sviluppatore mobile e web}
        {Softfobia}
        {Sviluppate e mantenute diverse app mobile e web, in progetti di diversa portata e per clienti di diverso tipo. In tali progetti ho avuto spesso modo di lavorare usando metodologia Agile, con comunicazioni costanti con il cliente.\\
        In ambito mobile ho avuto modo di lavorare sia sui framework nativi di Android e iOS, sia sfruttando tecnologie ibride come Ionic o Flutter, cercando di mantenere sempre un occhio critico sui pattern architetturali più adatti al singolo caso.}
\end{cvtable}


\cvsection{Titoli di studio}
\begin{cvtable}[2]
	\cvitem{2015 - 2019}
        {Laurea triennale in Informatica}
        {Università degli Studi di Cagliari}
        {Tesi relativa all'ambito della sentiment analysis.\\ Correlazione fra il sentiment dell'utenza di Stocktwits, ottenuto dall'analisi dei twit, e l'effettivo andamento dell'indice SP500.}
    \cvitem{2014}
        {Diploma di Perito tecnico, specializzazione Informatica}
        {ITIS Giua}
        {\\}
\end{cvtable}

\cvsection{Competenze Informatiche}
\begin{cvtable}[2.5]
    \cvitem{\texttt{UIKit\\ SwiftUI}}
        {App iOS}
        {}
        {Sviluppo e manutenzione app, implementazione di librerie, framework e SDK. Buona conoscenza dei più famosi pattern architetturali, di UIKit e SwiftUI, e di Combine.}
        
    \cvitem{\texttt{Flutter\\ Dart}}
        {Frontend mobile app}
        {}
        {Partecipato al programma di formazione interno ``mobile academy'' sullo sviluppo di app mobile basate su Flutter. Conoscenza di base del pattern Provider/Riverpod.}

    \cvitem{\texttt{Angular\\ Ionic\\HTML5+CSS3}}
        {Frontend mobile/web}
        {}
        {Tecnologie utilizzate su più progetti per la creazione e la manutenzione di app web e mobile, utilizzando anche diversi plugin di terze parti.}
    
        \cvitem{\texttt{Java\\ Kotlin}}
        {App Android}
        {}
        {Sviluppo e manutenzione app, implementazione di librerie, framework e SDK.}
    
    \cvitem{\texttt{Python}}
        {Script per raccolta e analisi dati inerenti la tesi di laurea, progetti personali}
        {}
        {Analisi e rappresentazione di dati basati su twit finanziari, raccolti tramite API dal sito StockTwits. Linguaggio usato anche per creare bot su Telegram e altri side-project.}
    
    \cvitem{\texttt{PHP}}
        {Gestione siti web}
        {}
        {A livello professionale: sviluppo e manutenzione di progetti basati su Drupal.\\Utilizzato anche durante la carriera scolastica e universitaria.}
    
    \cvitem{\texttt{C\# + Unity}}
        {Esame di Video Game Design}
        {}
        {Sviluppato, in team, un puzzle game 3D in prima persona, basato sulla sopravvivenza e la risoluzione di enigmi per avanzare di livello.}
    
    \cvitem{Altri}
        {Rust, C, C++, YAML, LaTex}
        {}
        {Linguaggi utilizzati al fine di superare vari esami, progetti personali o la stesura di questo Curriculum Vitae: conoscenze di base.}
\end{cvtable}

\cvsignature

\let\thefootnote\relax\footnotetext{Autorizzo il trattamento dei dati personali contenuti nel mio curriculum vitae in base all’art. 13 del D. Lgs. 196/2003 e all’art. 13 GDPR 679/16.}

\end{document}
